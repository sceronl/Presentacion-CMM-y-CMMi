\documentclass{beamer}
\usepackage[spanish]{babel}
\usepackage[utf8]{inputenc}
\usepackage{beamerthemeshadow}

\title{CMM y CMMi}
\author{Sebastián Cerón, Ignacio Briones, Fernando Rubilar}
\date{\today}

\begin{document}
\frame{\titlepage}

\section[Índice]{}
\frame{\tableofcontents}
\section{CMM}
\subsection{Que es}

\frame
{
  \frametitle{CMM}
  %Va mostrando un ítem por transparencia
  \begin{itemize}
    \item<1-> Modelo de Capacidad y Madurez o CMM (Capability Maturity Model), es un modelo de evaluación de los procesos de una organización. 
    \item<2-> Fue desarrollado inicialmente para los procesos relativos al software por la Universidad Carnegie-Mellon para el SEI (Software Engineering Institute).
    \item<3-> El SEI es un centro de investigación y desarrollo patrocinado por el Departamento de Defensa de los Estados Unidos de América y gestionado por la Universidad Carnegie-Mellon. CMM es una marca registrada del SEI.
  \end{itemize}
}

\frame
{
  \frametitle{CMM}
  %Va mostrando un ítem por transparencia
  \begin{itemize}
    \item<1-> A partir de noviembre de 1986 el SEI, a requerimiento del Gobierno Federal de los Estados Unidos de América, desarrolló una primera definición de un modelo de madurez de procesos en el desarrollo de software, que se publicó en septiembre de 1987. Este trabajo evolucionó al modelo CMM o SW - CMM (CMM for Software), cuya última versión (v1.1) se publicó en febrero de 1993.
  \end{itemize}
}

\subsection{Que hace}

\frame
{
  \frametitle{CMM}
  %Va mostrando un ítem por transparencia
  \begin{itemize}
    \item<1-> Este modelo establece un conjunto de prácticas o procesos clave agrupados en Áreas Clave de Proceso (KPA - Key Process Area). Para cada área de proceso define un conjunto de buenas prácticas que habrán de ser:
    \item<2-> Definidas en un procedimiento documentado.
    \item<3-> Provistas (la organización) de los medios y formación necesarios.
    \item<4-> Ejecutadas de un modo sistemático, universal y uniforme (institucionalizadas).
    \item<5-> Medidas.
    \item<6-> Verificadas.
  \end{itemize}
}

\frame
{
  \frametitle{CMM}
  %Va mostrando un ítem por transparencia
  \begin{itemize}
    \item<1-> A su vez estas Áreas de Proceso se agrupan en cinco "niveles de madurez", de modo que una organización que tenga institucionalizadas todas las prácticas incluidas en un nivel y sus inferiores, se considera que ha alcanzado ese nivel de madurez. Los niveles son:
    \item<2-> {\Large 1 - Inicial.} Las organizaciones en este nivel no disponen de un ambiente estable para el desarrollo y mantenimiento de software. Aunque se utilicen técnicas correctas de ingeniería, los esfuerzos se ven minados por falta de planificación. El éxito de los proyectos se basa la mayoría de las veces en el esfuerzo personal, aunque a menudo se producen fracasos y casi siempre retrasos y sobrecostes. El resultado de los proyectos es impredecible. 
  \end{itemize}
}

\frame
{
  \frametitle{CMM}
  %Va mostrando un ítem por transparencia
  \begin{itemize}
    \item<1->{\Large  2 - Repetible.} En este nivel las organizaciones disponen de unas prácticas institucionalizadas de gestión de proyectos, existen unas métricas básicas y un razonable seguimiento de la calidad. La relación con subcontratistas y clientes está gestionada sistemáticamente.

    \item<2-> {\Large 3 - Definido.} Además de una buena gestión de proyectos, a este nivel las organizaciones disponen de correctos procedimientos de coordinación entre grupos, formación del personal, técnicas de ingeniería más detallada y un nivel más avanzado de métricas en los procesos. Se implementan técnicas de revisión por pares (peer reviews).
  \end{itemize}
}

\frame
{
	\frametitle{CMM}
	\begin{itemize}
	    \item<1-> {\Large 4 - Gestionado.} Se caracteriza porque las organizaciones disponen de un conjunto de métricas significativas de calidad y productividad, que se usan de modo sistemático para la toma de decisiones y la gestión de riesgos. El software resultante es de alta calidad.

    	\item<2->{\Large  5 - Optimizado.} La organización completa está volcada en la mejora continua de los procesos. Se hace uso intensivo de las métricas y se gestiona el proceso de innovación.
	\end{itemize}
}

\frame
{
	\frametitle{CMM}
	\begin{itemize}
	    \item<1-> Cada nivel a su vez cuenta con un número de áreas
de proceso que deben lograrse. El alcanzar estas áreas o estadios se detecta
mediante la satisfacción o insatisfacción de varias metas claras y cuantificables. 

		\item<2-> Con la excepción del primer nivel, cada uno de los restantes Niveles de Madurez está
compuesto por un cierto número de Áreas Claves de Proceso, conocidas a través de
la documentación del CMM por su sigla inglesa: KPA.


    	\item<3-> Cada KPA identifica un conjunto de actividades y prácticas interrelacionadas, las
cuales cuando son realizadas en forma colectiva permiten alcanzar las metas
fundamentales del proceso. Las KPAs pueden clasificarse en 3 tipos de proceso:
Gestión, Organizacional e Ingeniería.
	\end{itemize}
}

\frame
{
  \frametitle{CMM}
  %Va mostrando un ítem por transparencia
  \begin{itemize}
    \item<1-> Las prácticas que deben ser realizadas por cada Area Clave de Proceso están organizadas en {\Large 5 Características Comunes}, las cuales constituyen propiedades que indican si la implementación y la institucionalización de un proceso clave es efectivo, repetible y duradero.

    \item<2-> Estas 5 características son: i)Compromiso de la realización, ii) La capacidad de realización, iii) Las actividades realizadas, iv) Las mediciones y el análisis, v) La verificación de la implementación.

  \end{itemize}
}

\subsection{Entonces}

\frame
{
  \frametitle{CMM}
  %Va mostrando un ítem por transparencia
  \begin{itemize}
    \item<1-> Las organizaciones que utilizan CMM para mejorar sus procesos disponen de una guía útil para orientar sus esfuerzos. Además, el SEI proporciona formación a evaluadores certificados (Lead Assesors) capacitados para evaluar y certificar el nivel CMM en el que se encuentra una organización. Esta certificación es requerida por el Departamento de Defensa de los Estados Unidos, pero también es utilizada por multitud de organizaciones de todo el mundo para valorar a sus subcontratistas de software.
    \item<2-> Se considera típico que una organización dedique unos 18 meses para progresar un nivel, aunque algunas consiguen mejorarlo. En cualquier caso requiere un amplio esfuerzo y un compromiso intenso de {\Large la dirección.}
  \end{itemize}
}

\frame
{
  \frametitle{CMM}
  %Va mostrando un ítem por transparencia
  \begin{itemize}
    \item<1-> Como consecuencia, muchas organizaciones que realizan funciones de factoría de software o, en general, outsourcing de procesos de software, adoptan el modelo CMM y se certifican en alguno de sus niveles. Esto explica que uno de los países en el que más organizaciones certificadas exista sea India, donde han florecido las factorías de software que trabajan para clientes estadounidenses y europeos.

    \item<2-> A partir de 2001, en que se presentó el modelo CMMI, el SEI ha dejado de desarrollar el SW-CMM, cesando la formación de los evaluadores en diciembre de 2003, quienes dispondrán hasta fin de 2005 para reciclarse al CMMI. Las organizaciones que sigan el modelo SW-CMM podrán continuar haciéndolo, pero ya no podrán ser certificadas a partir de fin de 2005.
  \end{itemize}
}

\section{CMMi}
\subsection{Descripcion}
\frame
{
  \frametitle{CMMi}
  	\begin{itemize}
  	\item<1-> Integración de modelos de madurez de capacidades o Capability maturity model integration (CMMI) es un modelo para la mejora de procesos que proporciona a las organizaciones los elementos esenciales para procesos eficaces.
  	\item<2-> Las mejores prácticas CMMI se publican en los documentos llamados {\Large modelos}.
  	\item<3-> Modelos de CMMI: Desarrollo, Adquisición y Servicios.
  	\end{itemize}
}

\frame
{
  \frametitle{CMMi}
  	\begin{itemize}
  	\item<1-> Desarrollo
  	\item<2-> Adquisición: En él se tratan la gestión de la cadena de suministro, adquisición y contratación externa en los procesos del gobierno y la industria.
  	\item<3-> Servicios: Gestionar, establecer y entregar Servicios.
  	\end{itemize}
}

\frame
{
  \frametitle{CMMi}
  	\begin{itemize}
  	\item<1-> CMMI se desarrolló para facilitar y simplificar la adopción de varios modelos de
forma simultánea, y su contenido integra y da relevo a la evolución de sus
predecesores:

  	\item<2-> CMM-SW (CMM for Software).
	\item<3-> SE-CMM (Systems Engineering Capability Maturity Model).
    \item<4-> IPD-CMM (Integrated Product Development).

  	\end{itemize}
}

\frame
{
  \frametitle{CMMi}
  	\begin{itemize}
  	\item<1-> El cuerpo de conocimiento disponible en CMMI incluye:

  	\item<2-> Systems engineering (SE)(Ingeniería de sistemas): Esto es respecto a servidores, etc
	\item<3-> Software engineering (SW)(ingeniería de software): El ''codeo'' es decir lo respecto al desarrollo del codigo
	\item<4-> Integrated product and process development (IPPD)(Desarrollo integrado de productos y procesos):  esto se refiere a como se integra con los demas productos de al organizacion
	\item<5-> Supplier sourcing (SS)(Proveedores): Respecto a como se relaciona con los proveedores y como se abastecen.
\end{itemize}
}


\frame
{
  \frametitle{CMMi}
  	\begin{itemize}
  	\item<1-> Las organizaciones no pueden ser certificadas CMMi, por el contrario una organizacion es Evaluada
  	\item<2-> Ejemplo metodo de evaluacion SCAMPI (Calificaciones de 0 a 5)
  	\item<3-> La organizacion se evalua segun criterios de Madurez y a la vez esta misma puede escoger areas de procesos o niveles de madurez obteniendo asi el perfil de capacidad de la organizacion
  	\end{itemize}
}

\frame
{
  \frametitle{CMMi}
  	\begin{itemize}
  	\item<1-> Las Evaluaciones son realizadas normalmente por una o más de las siguientes razones:
  	\item<2-> Para determinar que tan bien los procesos de la organización se comparan con las mejores prácticas CMMI y determinar qué mejoras se pueden hacer.
  	\item<3-> Para informar a los clientes externos y proveedores acerca de que tan bien los procesos de la organización se comparan con las mejores prácticas CMMI.
  	\item<4-> Para cumplir los requisitos contractuales de uno o más clientes.
  	\end{itemize}
}

\frame
{
  \frametitle{CMMi}
  	\begin{itemize}
\item<1-> El modelo CMMI v1.2 (CMMI-DEV) contiene las siguientes 22 áreas de proceso:
\item<1-> Análisis de causalidad y solución
\item<1-> Configuration Management
\item<1-> Decisión de Análisis y Resolución
\item<1-> Proyecto Integrado de Gestión
\item<1-> Medición y Análisis
\item<1-> Innovación organizacional y Despliegue
\item<1-> Definición de procesos organizacionales
\item<1-> Enfoque en procesos organizacionales
\item<1-> Rendimiento de procesos organizacionales
\item<1-> Entrenamiento organizacional

  	\end{itemize}
}

\frame
{
  \frametitle{CMMi}
  	\begin{itemize}
  	
\item<1-> Vigilancia y Control de proyectos
\item<1-> Planificación de proyectos
\item<1-> Proceso y aseguramiento de calidad del producto
\item<1-> Integración de Producto
\item<1-> Gestión de proyectos Cuantitativos
\item<1-> Gestión de requerimientos
\item<1-> Requerimientos de Desarrollo
\item<1-> Gestión de Riesgos
\item<1-> Gestión de Proveedores
\item<1-> Solución
\item<1-> Validación
\item<1-> Verificación
  	\end{itemize}
}

\subsection{Por que usar, es el mejor, Comparacion}

\frame
{
  \frametitle{Por que debe usarse}
  	\begin{itemize}
  	\item<1-> Un modelo nos permite comprender los elementos especificos de las organizaciones y ayuda a formular y a hablar de lo que hay que mejorar.
  	\item<2-> Un modelo Ofrece las Siguientes Ventajas:
  	\item<3-> Proporciona un marco de trabajo
  	\item<4-> Aporta años de experiencias
  	\item<5-> Suele tener el respaldo de instructores y/o consultores
  	\item<6-> Puede proporcionar un standar para salvar las discrepancias
  	\end{itemize}
}

\frame
{
  \frametitle{Beneficios CMMi}
  	\begin{itemize}
  	\item<1-> La gestion e ingenieria estan mas explicitamente enlazadas para los objetivos del negocio.
  	\item<2-> Incorporar la experiencia adquirida en otras zonas de las mejores practicas.
  	\item<3-> Aplicar practicas de alta madurez mas robustas
  	\item<4-> Cumplir lo mas completamente con las normas ISO
  	\end{itemize}
}

\frame
{
  \frametitle{Comparacion de Madurez}
  	\begin{itemize}
  	\item<1-> Empresa Inmadura:
  	\item<2-> Apaga fuegos.
  	\item<3-> Tiene poco recursos propios.
  	\item<4-> Tiene exitos gracias a los {\Large HEROES}.
	\item<5-> Las planificaciones son poco realistas.
	\item<6-> Los plazos de entrega son impredecibles.  	
  	\end{itemize}
}

\frame
{
  \frametitle{Comparacion de Madurez}
  	\begin{itemize}
  	\item<1-> Empresa Madura:
  	\item<2-> Tienen procesos definidos.
  	\item<3-> Tienens responsabilidades definidas.
  	\item<4-> Resultados predecibles.
  	\item<5-> Entrega la cantidad esperada.
  	\item<6-> Cumple plazos de entrega.
  	\item<7-> Reconoce las mejoras.
  	\item<8-> satisface a los clientes
  	\end{itemize}
}

\frame
{
  \frametitle{Conclusiones}
  	\begin{itemize}
  	\item<1-> {\Large Conclusiones y Preguntas.}
	\item<2-> El certificarnos o evaluarnos bajo CMM y CMMi nos da una confiabilidad ante el cliente de que vamos a responder de acuerdo a lo requerido
  	\end{itemize}
}

\end{document}